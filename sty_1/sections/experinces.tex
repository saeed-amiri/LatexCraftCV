\section{Experiences}

%====================
% EXPERIENCE A
%====================
\vspace{0.33em}
\begin{timelineentry}{0.4em}{-5pt}{4pt}{MLOps}
  \subsection{Ongoing Training: Machine Learning Engineer (Data Science \& MLOps) \hfill Present --- (Expected Completion) Mar 2026}
  \vspace{0.33em}
  \subtext{Liora (a.k.a. DataScientest) \hfill Université Paris 1 Panthéon-Sorbonne Certificate (Full-time, Online)}
\end{timelineentry}

\vspace{-0.83em}
\begin{zitemize}
\item \textbf{Production-oriented training: data science, machine learning, and MLOps:}
\begin{itemize}
    \item Focus on real-world constraints (scale, imbalance, deployment, monitoring)
    \item Applying a variety of supervised and unsupervised ML methods
    \item MLOps lifecycle management, deployment, and monitoring
    \item Cloud (AWS), Docker, Kubernetes, Airflow, and FastAPI
    \item API security, logging, Prometheus, and Grafana
\end{itemize}
\end{zitemize}

\vspace{0.33em}

%====================
% EXPERIENCE B
%====================
% \vspace{0.33em}
\vspace{-0.63em}
\begin{timelineentry}{0.4em}{-5pt}{4pt}{postdoc}
\subsection{Postdoctoral Researcher \hfill Feb 2022 --- Jan 2025}
\vspace{0.33em}
\subtext{University of Bremen, Hybrid Materials Interfaces Group \hfill Bremen, Germany}
\end{timelineentry}

\vspace{-0.63em}
\begin{zitemize}
  \item \textbf{Developing first-principles numerical models for complex experimental systems (GROMACS, LAMMPS)}:
  \begin{itemize}
    \item Designed and implemented multi-scale simulation workflows to study Silica nanoparticles Laden interfaces.
    \item Derived numerical model, bridging continuum theory with simulations to model the \textbf{electrostatic double layer}.
    \item Collaborated closely with experimentalists to validate simulation results against synchrotron \textbf{X-ray reflectometry}.
    \item Developed automation tools and generalized Python/Bash pipelines for large-scale job submission on \textbf{HPC} clusters.
  \end{itemize}
  \vspace{0.33em}
\end{zitemize}


% \vspace{0.33em}
%====================
% EXPERIENCE C
%====================
\begin{timelineentry}{0.4em}{-5pt}{7pt}{phd}
  \subsection{{Doctoral Researcher \hfill May 2018 --- Oct 2021}}\vspace{0.33em}
  \subtext{University of Göttingen, Institut für Materialphysik \hfill Göttingen, Germany}\vspace{0.33em}
\end{timelineentry}

\vspace{-1.03em}
\begin{zitemize}
  \item \textbf{Modeled and simulated complex atomic-scale friction systems}
  \begin{itemize}
   \item Developed models within the \textbf{LAMMPS} framework to simulate complex nanotribological phenomena.
    \item Built hybrid \textbf{Python/C} toolsets for automated parameter space exploration and high-fidelity data analysis.
    \item Scaled simulations on \textbf{HPC clusters}, optimizing MPI-based parallelization to handle large-scale atomic systems efficiently.
  \end{itemize}
\end{zitemize}
\break

\begin{timelineentry}{0.4em}{-5pt}{4pt}{masters}
\subsection{Graduate Researcher (Master's) \hfill Sep 2013 --- Sep 2016}
\vspace{0.33em}
\subtext{Damghan University\hfill Damghan, Iran}
\end{timelineentry}

\vspace{-0.63em}
\begin{zitemize}
  \item \textbf{First-principles characterization of low-dimensional materials}:
  \begin{itemize}
    \item Study electronic structure models for monolayer materials using \textbf{DFT} frameworks (\textbf{SIESTA} and \textbf{Quantum Espresso}).
    \item Utilized \textbf{Python, Fortran} for automated data extraction and post-processing of simulation outputs.
    \item Investigated the influence of strain and defect engineering on monolayer stability.
    \item Characterized structural, electronic, magnetic properties of 2D materials
  \end{itemize}
\end{zitemize}